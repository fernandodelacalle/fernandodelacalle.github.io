\documentclass[10pt]{scrartcl}
\usepackage{lmodern}
\usepackage[nochapters]{classicthesis} % Use the classicthesis style for the style of the document
\usepackage[left=1.7in,top=0.9in,right=0.75in,bottom=0.6in]{geometry}

\reversemarginpar % Move the margin to the left of the page 
\newcommand{\MarginText}[1]{\marginpar{\raggedleft\itshape\small#1}} % New command defining the margin text style
\setlength\parindent{0pt}
\usepackage{marginnote}
\renewcommand*{\marginfont}{\bfseries\fontfamily{lmss}}
\usepackage[LabelsAligned]{currvita} % Use the currvita style for the layout of the document
\renewcommand{\cvheadingfont}{\LARGE\color{Maroon}} % Font color of your name at the top
\usepackage{hyperref} % Required for adding links   and customizing them
\hypersetup{colorlinks, breaklinks, urlcolor=, linkcolor=} % Set link colors

%\newlength{\datebox}\settowidth{\datebox}{Spring 2011} % Set the width of the date box in each block
%\newcommand{\NewEntry}[3]{\noindent\hangindent=2em\hangafter=0 \parbox{\datebox}{\small \textit{#1}}\hspace{1.5em} #2 #3 % Define a command for each new block - change spacing and font sizes here: #1 is the left margin, #2 is the italic date field and #3 is the position/employer/location field
%\vspace{0.5em}} % Add some white space after each new entry
%\newcommand{\Description}[1]{\hangindent=2em\hangafter=0\noindent\raggedright\footnotesize{#1}\par\normalsize\vspace{1em}} % Define a command for descriptions of each entry - change spacing and font sizes here
\usepackage{marvosym}
%----------------------------------------------------------------------------------------

\begin{document}
\date{}
\thispagestyle{empty} % Stop the page count at the bottom of the first page

%----------------------------------------------------------------------------------------
%   NAME AND CONTACT INFORMATION SECTION
%----------------------------------------------------------------------------------------

\begin{cv}{\spacedallcaps Fernando de la Calle Silos}\vspace{0.5em} % Your name

%------------------------------------------------
{\footnotesize
\Telefon  \hspace{0.4em} (+34) 677 124 633 \hspace{0.4em} \\ \hspace{0.4em}  \Letter   \hspace{0.4em} \href{mailto:fsilos@tsc.uc3m.es}{\texttt{fsilos@tsc.uc3m.es}} \hspace{0.4em} \\   \hspace{0.4em}  \Mundus \hspace{0.4em} \url{http://www.tsc.uc3m.es/~fsilos}
}


\vspace{1em} % Extra space between major sections

%----------------------------------------------------------------------------------------
%   PUBLICATIONS
%----------------------------------------------------------------------------------------

\vspace{1em}

{Ph.D, Multimedia and Communications.}  \hfill  Universidad Carlos III de Madrid\marginnote{Education}\\
\textit{Jun 2013 -- Sep 2017} \\ 
PhD dissertation: \textit{Bio-motivated Features and Deep Learning for Robust Speech Recognition}. 
\vspace{0.5em}

{M.Sc, Multimedia and Communications.}   \hfill Universidad Carlos III de Madrid\\ \textit{Sep 2012 -- Jun 2013} %GPA Overall:  3.4   %GPA 8.5 on a 10.0 scale.
\vspace{0.5em}

{B.Sc, Audiovisual System Engineering.}  \hfill Universidad Carlos III de Madrid\\ \textit{Sep 2008 -- Jun 2012}\\ %GPA Overall:  3.12  %GPA 7.8  on a 10.0 scale.
BSc dissertation: \textit{Event Recognition in Crowded Scenes}. 
\vspace{1em}


{\color{Maroon} Data Scientist} \hfill  Treelogic\marginnote{Experience}\\
\textit{Jan 2018 -- Present}  \hfill IDI Department\vspace{0.5em}

Deep learning researcher in computer vision related topics, applying convolutional neural networks to object detection and classification in real time. Deep reinforcement learning applied to real environments. 
\vspace{1em}

{\color{Maroon} Assistant Professor} \hfill  Universidad Carlos III de Madrid\\
\textit{Sep 2017 -- Present}   \hfill Signal Theory and Communications Department\vspace{0.5em}

I have been teaching subjects such as: Machine learning, Data Processing, Digital Image Processing, Environmental Noise Control, Digital Audio Processing and Algorithms for Information Retrieval.

I have four BSc and MSc students under my supervision doing their final degree projects.
\vspace{1em}


{\color{Maroon} Visiting Researcher} \hfill  Carnegie Mellon University\\
\textit{Aug 2015 -- Jan 2016}  \hfill Computer Science Department\vspace{0.5em}

I collaborated with Prof. Richard Stern to develop signal processing and machine learning algorithms for robust speech recognition.
\vspace{1em}

{\color{Maroon} PhD  Candidate}  \hfill  Universidad Carlos III de Madrid\\
\textit{Jun 2013 -- Sep 2017}  \hfill  Multimedia Processing Group\vspace{0.5em}

My PhD was totally funded by AIRBUS Defense and Space.  
It was focused on improving the performance of the current  speech recognition systems in noisy and stressful environments by employing novel signal processing techniques and machine learning algorithms (mainly deep learning).
An automatic speech recognition system was developed and installed in the ground control station providing speech control facilities for unmanned aerial vehicles.



\vspace{1em}

{\color{Maroon} Researcher} \hfill   Universidad Carlos III de Madrid\\
\textit{Sep 2011 -- Jan 2018} \hfill Multimedia Processing Group \vspace{0.5em}

During my time as a researcher I participated in the following research projects:

- Development of computer vision algorithms for road safety and other applications.

- {Context-Aware Automatic Speech Recognition under Cognitive Stress aided by Multimodal Biometric Detection} funded by Airbus Defense and Space.

- {PROSAVE2-Research project in advanced systems for a more eco-efficient aircraft.} funded by the European Aeronautic Defense and Space Company (EADS). Where I worked on the implementation of the tracking algorithm of the aerial refueling boom system of the Airbus A330 MRTT.\\	
- {Prospective and algorithms design for video coding} funded by  Procesamiento Digital y Sistemas S.L. (PRODYS). 
	
- {Annotation, indexing and coding of user generated content}, funded by the Ministry of Science and Innovation of Spain.	




\vspace{1.5em}

{\bfseries\fontfamily{lmss}Academic Awards}\marginnote{Recognition} 

\vspace{0.5em}
{\color{Maroon}Outstanding Thesis Award} of year 2017 in the Multimedia and Communications PhD Program, Universidad Carlos III de Madrid, Spain.

\vspace{0.5em}
{\color{Maroon}Best Academic Record Graduation Award} of class 2008-2012 in the BSc of Audiovisual System Engineering, Universidad Carlos III de Madrid, Spain.

\vspace{1em}

{\bfseries\fontfamily{lmss}Research Awards}

\vspace{0.5em}
Best indexed 2016 JCR journal publication by the Spanish Thematic Network on Speech Technology  (RTTH) for the paper \textit{Morphologically-filtered power-normalized cochleograms as robust, biologically inspired features for ASR}.

\vspace{0.5em}
Best poster presentation of the Spanish Thematic Network on Speech Technology (RTTH) Summer School in 2013 and 2015.



\vspace{2em}
{\bfseries\fontfamily{lmss}}\marginnote{Other Formation} 

6th Lisbon Machine Learning School. The school covers a range of machine learning topics, from theory to practice, that are important in solving natural language processing problems.  \textit{July, 2016}	

\vspace{0.5em}
RTTH Summer Schools: Speech Technology - A Deep Learning Perspective, \textit{July, 2015} and Speech Technology Evaluation, \textit{July, 2013.}
	
%Spanish Thematic Network on Speech Technology  (RTTH) Summer School. Five-day intensive course covering the most relevant aspects related to the evaluation of speech technology. Vigo, Spain.  \textit{July, 2013.}
\vspace{0.5em}
Fundamentals of iOS Programming and Advanced iOS Programming bootcamps taught by  \href{https://keepcoding.io}{Keepcoding}. \textit{September and April, 2012.} %These 2 bootcamps cover 100\% of the \textit{``Cocoa Core Competencies''} document by Apple.



\vspace{2em}
{\bfseries\fontfamily{lmss}}\marginnote{Skills} 

\textbf{Software Skills:} Extensive programming experience in Python and Matlab, working experience in C/C++, Objective-C, Java, iOS and Android app development. Experience in scientific computing in a clustered environment.

\textbf{Machine learning:}  Experience with state of the art algorithms, currently focus on deep learning. Knowledge of GPU acceleration for machine learning (CUDA). Advanced user of Tensorflow and Theano. Speech recognition tools: kaldi and sphinx. 

\textbf{Others:} Experience in research collaborations between academia and industry.

\textbf{Multilingual:} Fluent in English, Spanish as native language.



\vspace{1.5em}

{\bfseries\fontfamily{lmss} Journal}
\marginnote{Publications} 
\vspace{0.5em}

{F. de-la-Calle-Silos} and Richard M. Stern. \textit{`Synchrony-Based Feature Extraction for Robust Automatic Speech Recognition,''} in  \textit{ IEEE Signal Processing Letters, vol. 24, no. 8, pp. 1158, Aug. 2017.}

\vspace{0.5em}

{F. de-la-Calle-Silos}, F.J. Valverde-Albacete, A. Gallardo-Antol\'in, C. Pela\'ez-Moreno. \textit{`Morphologically filtered power normalized cochleograms as robust, biologically inspired features for ASR,''} in  \textit{ IEEE/ACM Transactions on Audio, Speech, and Language Processing, vol. 23, no. 11, pp. 2070-2080, Nov. 2015.}

\vspace{1em}
{\bfseries\fontfamily{lmss} Selected Conferences}
\vspace{0.5em}

{F. de-la-Calle-Silos}, A. Gallardo-Antol\'in, C. Pela\'ez-Moreno. \textit{``An Analysis of Deep Neural Networks in Broad Phonetic Classes for Noisy Speech Recognition''}, \textit{Advances in Speech and Language Technologies for Iberian Languages.  Communications in Computer and Information Science, Springer 2016.} 
\vspace{0.5em}

%{F. de-la-Calle-Silos}, F.J. Valverde-Albacete, A. Gallardo-Antol\'in, C. Pela\'ez-Moreno. \textit{`Preliminary experiments on the robustness of biologically motivated features for DNN-based ASR''}, \textit{4th International Work Conference on Bioinspired Intelligence (IWOBI 15)}, \textit{June 2015.}	
%\vspace{0.5em}	

{F. de-la-Calle-Silos}, A. Gallardo-Antol\'in, C. Pela\'ez-Moreno. \textit{``Deep Maxout Networks applied to Noise-Robust Speech Recognition''}, \textit{Advances in Speech and Language Technologies for Iberian Languages.  Communications in Computer and Information Science, Springer 2014.} 	
\vspace{0.5em}	

{F. de-la-Calle-Silos}, F.J. Valverde-Albacete, A. Gallardo-Antol\'in, C. Pela\'ez-Moreno. \textit{``ASR Feature Extraction with Morphologically-Filtered Power-Normalized Cochleograms''}, \textit{Anual Conference of the International Speech Communication Association (INTERSPEECH)}, Singapore, \textit{September 2014.}
\vspace{0.5em}

{F. de-la-Calle-Silos}, I. Gonz\'alez-D\'iaz,  F. D\'iaz-de-Mar\'ia. \textit{``Mid-Level Feature Set for Specific Event and Anomaly Detection in Crowded Scenes''}, \textit{IEEE International Conference of  Image Processing (ICIP)}, Melbourne, Australia, \textit{September 2013.}


%\vspace{5 mm}
%\begin{flushright} 
%	\tiny Apr, 2018
%\end{flushright}
%\end{cv}
\end{document}